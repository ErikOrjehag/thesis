\chapter{Related work}\label{cha:relatedwork}


In this chapter some of the important contributions of previous research papers are summarized. Firstly a list of papers in the field of unsupervised depth and ego motion prediction is presented. Secondly a paper on unsupervised feature point prediction. Thirdly a paper on unsupervised geometric consensus maximization that was implemented in this thesis. Finally a paper on combining depth and feature point prediction, albeit using classical consensus maximization that is not learned by a neural network.

\section{Unsupervised depth and ego motion prediction}
\label{sec:rldepth}

All the papers in this section are discussed in the order of publication to show a timeline of progress in the field of depth and ego motion prediction.

%\section{Unsupervised Monocular Depth Estimation with Left-Right Consistency}
%\label{sec:relwork:leftright}

In this paper\cite{leftright} the authors present MonoDepth, with an implementation available in Tensorflow on github. In this work the depth is predicted using an encoder-decoder type network, but the relative motion between frames is not estimated at all. The KITTI dataset provides stereo image pairs which are used during training, and the relative transformation between the left and right cameras is known. Using only the left image as input to the network both disparity maps for the left and right images are predicted. The two disparity maps are used to project the left image into the view of the right image and vice versa. This can be seen as a precursor to the papers discussed later which uses only a monocular camera and several frames over time to train a depth predicting network and pose predicting network jointly. The L1 norm of the per pixel photometric error as well as \abbrSSIM\cite{ssim} are computed and added to the loss. An additional loss term encourages the left disparity to be equal to the right disparity projected into the left camera viewpoint. Because the photometric error does not work well on low textured regions an edge aware smoothess term is added to propagate the depth values from nearby areas in the disparity map. The method produces metrically accurate results because the baseline and focal length of the cameras are known.

%\section{Unsupervised Learning of Depth and Ego-Motion from Video}
%\label{sec:relwork:unego}

In this paper\cite{sfmlearner} the authors present SfMLearner with an official implementation in Tensorflow on github. Contrary to the previous paper\cite{leftright} only a monocular sequence of images from the KITTI dataset is used during training. In the stereo case the relative pose between the left and right cameras was known, but with this monocular dataset the pose between subsequent frames is unknown. The authors train a pose predicting and depth predicting network simultaneously with a joint loss function to solve this problem.

During training 3 subsequent frames are considered at a time. The frame $I_{t-1}$ and frame $I_{t+1}$ are called the source frames and the frame $I_t$ is called the target frame. The target frame is input to the depth network which estimates a disparity map. The two source frames are fed through a pose estimating network one after each other together with the target frame to find the relative transformations $T_{t	\rightarrow t-1}$ and $T_{t	\rightarrow t+1}$.

The authors add an explainability mask to the photometric error term to account for errors in the model. The view synthesis formulation implicitly assumes that the scene does not contain moving objects, that there are no occlusions between the target and source frames, and that the surfaces are Lambertian so that the photo-consistency error of \abbrRGB values is meaningful. In order to predict the explainability mask an additional \abbrCNN is used. The network has no explicit supervisory signal but is encouraged to be non-zero with an regularization term using a cross-entropy loss with a constant label 1 at each pixel location. This makes the network minimize the view synthesis objective but is allowed some slack due to factors not considered by the model. In later work it was shown that the explainability mask does not help to improve results that much and is often ignored.

To tackle the problem with textured areas and non Lambertian surfaces, a smoothess term is used. An edge aware smoothess term was not used like in previous work\cite{leftright} but a penalty on the second order gradient of the depth map was used instead. This unfortunately makes edges very fuzzy in the results compared to using the edge aware smoothness loss.

%\section{SuperDepth: Self-Supervised, Super-Resolved Monocular Depth Estimation}

% TA BORT?
In this paper\cite{superdepth} the authors train a depth estimating network on only stereo image pairs. After the depth predicting network has been trained the pose network from \cite{sfmlearner} is trained, using results from the already trained depth predicting network. This means that the depth and pose networks are not trained with a joint loss function but are instead trained separately. The main contributions from this paper is a new subpixel-convolution operation that super-resolve disparities from their low-resolution outputs, thereby replacing the up-sampling layers typically used in the disparity decoder network. The method additionally uses differentiable flip-augmentation to remove edge artifacts on the left and right edges of the depth map seen in previous work using stereo image pairs during training. To handle occluded pixels between the left and right images an occulsion regularization loss term is added to encourage background depths (low disparaty).

%\section{3D Packing for Self-Supervised Monocular Depth Estimation}

% TA BORT?
In this paper\cite{packnet} the authors present PacknetSfM. The main contribution is a new network architecture with packing and unpacking blocks replacing down and up sampling. The new packing blocks use space to depth transformations and 3d convolutions. The authors claim that the new packing and unpacking blocks are better at perserving resolution than standard down and upsampling. As the method is not adopted in later work, it is unclear if the new packing and unpacking approach is effective. The second contribution is a loss on the camera velocity that makes it possible for the monocular depth estimation to be metrically accurate. The velocity of the car is assumed to be known in the training dataset. The third contribution is a mask on pixels that do not change between frames. These pixels are assumed to be part of objects that are stationary with respect to the camera. Such pixels can occur for example if the car dashboard is visible in a frame, or other vehicles are moving at a similar speed nearby.

%\section{Digging Into Self-Supervised Monocular Depth Estimation}

In this paper\cite{monodepth2} the authors present MonoDepth2. They propose mainly two contributions. Firstly, instead of taking the average of the reprojection errors from all source frames given a target frame they use the minimum. This makes it so that if a feature is occluded in one source image but not in the other the errors will not be averaged together but instead the error from the source frame that is not occluded will be used. Secondly, instead of calculating the loss for each depth scale in the decoder, all the depth maps are up-sampled to the original target image size when computing the loss. This way a single pixel in the low resolution layer of the decoder will predict the depth of a patch of pixels in the originally sized input image.

\section{Unsupervised keypoint prediction}
%\section{UnsuperPoint: End-To-End Unsupervised Interest Point Detector and Descriptor}

Collecting ground truth data to train a neural network for keypoint detection is cumbersome. What constitutes a keypoint is hard to define clearly and consistently for a human annotator.

SuperPoint\cite{superpoint} uses a unsupervised approach to learn prediction of points and their descriptors. The network is first trained on  a synthetic dataset of simple geometric shapes. The pseudo ground truth points are defined as the corners and junctions in the synthetic images. This pre-trained network is then trained on "real" images in a siemese network setup using a method they call homographic adaptation. The two siamese siblings are fed images transformed by random but known homographies and the output from the networks are compared in the loss function. The output of SuperPoint is a heatmap where each pixel describes the "point-ness" of that pixel, and a descriptor map. To keep the model fast both the point and descriptor maps are predicted semi-dense grid,
one cell for each 8 pixel patch. The maps are up-sampled to the original image size using bicubic interpolation, and the descriptors are normalized.

UnsuperPoint\cite{unsuperpoint} is heavily inspired by SuperPoint but does not need to be pre-trained on a synthetic dataset, instead it is trained in one round of training on real images. It uses a similar method of using siamese networks and homography adaptation during traning, but the output from the network is different. The network uses regression of actual point position coordinates instead of a heat map, incorporates non-maximum suppression, predicts scores for each keypoint and a sparse descriptor map that is sampled and interpolated using the point positions.

\section{Unsupervised geometric consensus maximization}
%\section{Unsupervised Learning of Consensus Maximization for 3D Vision Problems}

Consensus maximization is an important strategy in 3D vision problems for robust geometric model estimation from measurements including outliers. The classical method of \textit{Random Sampling and Consensus}, or RANSAC\cite{ransac} for short, is widely popular with great success. But replicating the same generic behavior using supervised training of neural networks has proven difficult. Unsupervised methods have a huge potential to generalize to any unseen data distribution and are in this context very desirable. This paper\cite{consensus} introduces just such an unsupervised method of consensus maximization for 3D vision problems. Using the relationship between the set of inliers, and the subspace of polynomials representing the space of target transformations a model fitting cost can be can be defined without knowing the specific parameters of the geometric transformation. During learning the loss is defined in a way to learn the largest set of inliers with a low model fitting cost. The geometric model parameters can then be extracted from the polynomials after network has learnt to distinguish inliers and outliers.

\section{Combined depth and feature point detection}

In a paper\cite{keypointdepth} the authors combines the previously discussed MonoDepth2\cite{monodepth2} for depth prediction, but also adds keypoint learning from SuperPoint\cite{superpoint} into the pipeline. The researchers train the depth, keypoint and pose estimating networks jointly, making them benifit from each other and achieve state of the art performance.  However, the step that finds possible outlier keypoints, is not differentiable and is not trained jointy with the depth and keypoint networks. The step that calculates the model parameters of the geometric relationship between corresponding keypoints also requires an initial guess that is not differentiable.

