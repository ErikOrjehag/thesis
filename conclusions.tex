\chapter{Conclusions}\label{cha:conclusions}

The goal of this thesis has been to evaluate a few different unsupervised learning methods in the context of structure from motion. The techniques were combined in new ways and trained on new datasets not used in previous work.

\paragraph{\textbf{How well do the unsupervised methods from previous research work on new datasets not tested in the original papers?}} As we have seen, the depth prediction networks can learn to predict depth on the Lyft dataset, not used in previous work. The keypoint prediction network can be trained on images from the Kitti dataset. With some alterations to the consensus maximization network, to improve convergence, it can be trained on the output of the keypoint prediction network. This gives us more confidence that the 3 systems could be chained together and trained jointly in future work. The performance of each technique is detailed in the results chapter.

\paragraph{\textbf{What are the performance gains of combining different methods from recent research in monocular depth and ego motion prediction?}} Table \ref{table:experiments} lists all experiments conducted to compare different techniques in monocular depth and ego motion prediction. The biggest performance gains comes from using SSIM in the photometric error, and using the min() function to combine the per pixel loss across frames. Using a stationary pixel mask seems to be more effective compared to using a predicted explainability mask. Using the edge aware depth smoothess loss term does not improve the metrics much, but does give sharper depth maps upon visual inspection. The techniques with the least impact are depth map normalization and depth map upscaling.
\\
\\
We conclude that the goal of the project has been reached, and that there is a big potential in further research into this field.