\section{Evaluation}

This section describes the metrics used to evaluate depth, ego motion, feature point and consensus maximization predictions. These specific metrics where chosen because they are also used in the related works in this field, which makes the results comparable to other papers.

\subsection{Depth}

The depth error and accuracy metrics are sparse and only calculated for the pixels of which there exists a ground truth laser measurement in the dataset. The values are averaged over all laser measurements and frames in the test split of the dataset. An error of 0 and an accuracy of 1 is optimal, but can never be achieved in practice.

The depth is predicted relative to an unknown scale, and the ground truth is measured in meters. To alleviate this issue the depth predictions are scaled so that their median is the same as the median of the ground truth for each frame. This does not guarantee that the unit of the predictions is meters, but at least it makes the scales somewhat similar.

The error metrics used in the results chapter are referred to as \textit{Abs Rel}, \textit{Sq Rel}, \textit{RMSE} and \textit{RMSLE}.

The \textit{Abs Rel} error is based on \textit{MAE} which is the mean of the absolute errors which means that it has the same unit as the errors, and is conceptually quite easy to interpret.

\[
\textrm{MAE}=\frac{\sum^N_{n=1}{|y_n-\hat{y}_n|}}{N}
\]

Because the depth from the neural network is without unit and only predicted relative to an unknown scale, a variation on the \textit{MSE} metric called \textit{Abs Rel} is used.

\[
\textrm{Abs Rel}=\frac{\sum^N_{n=1}{\frac{|y_n-\hat{y}_n|}{y_n}}}{N}
\]

The \textit{MSE} metric is the mean of squared errors, for an unbiased estimator it represents the variance of the errors. 

\[
\textrm{MSE}=\frac{\sum^N_{n=1}{(y_n-\hat{y}_n)^2}}{N}
\]

Once again, because the depth is predicted relative to an unknown scale a variation on \textit{MSE} called \textit{Sq Rel} is used instead.

\[
\textrm{Sq Rel}=\frac{\sum^N_{n=1}{\frac{(y_n-\hat{y}_n)^2}{y_n}}}{N}
\]

The \textit{RMSE} metric is the square root of the mean of squared errors. For an unbiased estimator it represents the standard deviation of the errors. Because the errors are squared in the \textit{RMSE} metric it is more sensitive to outliers that for example \textit{MSE}.

\[
\textrm{RMSE}=\sqrt{\frac{\sum^N_{n=1}{(y_n-\hat{y}_n)^2}}{N}}
\]

The \textit{RMSLE} metric is similar to \textit{RMSE} but is useful because it penalizes large errors less when both the actual and predicted values are large.

\[
\textrm{RMSLE}=\sqrt{\frac{\sum^N_{n=1}{(\log{y_n}-\log{\hat{y}_n)}^2}}{N}}
\]

To measure the depth accuracy, in the range from 0 to 1, the following metric is used.

\[
a_{\gamma} = \frac{\sum^N_{n=1}{(\max(\frac{y_n}{\hat{y}_n}, \frac{\hat{y}_n}{y_n}) < \gamma)}}{N},\textrm{ for }\gamma \in \{1.25, 1.25^2, 1.25^3\}
\]

This should be interpreted as the ratio of predictions that is within the ratio of $ \gamma $ relative to the ground truth.

\subsection{Ego motion}

The error of the camera motion predictions are measured using \textit{RMSE}. But instead of taking the mean over all poses in a sequence, the alignment error is calculated part wise over a track length of only 5 poses. The final error is presented as the mean \textit{RMSE} of all parts. Each track part is transformed to have its first pose coincide with the world origo. This makes it so that the first pose in the track part of both the prediction and ground truth is the identity matrix $I \in \mathbb{R}^{4\times 4}$. Because the ego motion is predicted relative to an unknown scale and the ground truth is in meters, we scale each predicted track part to have similar scale to the ground truth track part.

\subsection{Keypoints}


\subsection{Consensus maximization}