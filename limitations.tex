\section{Handling model limitations}
\label{sec:modellimit}

In order to optimize using the photometric reprojecton error as the loss function two assumptions must hold. Firstly the scene must be static, meaning all objects in the scene must be still except the moving camera. Movement by cars and humans in the scene that is not due to the camera movement will cause problems. Secondly there must be photometric consistency between frames for the photometric error to make sense. This means that non lambertian surfaces, change in lighting, and change in exposure between frames will cause problems.

\paragraph{Explainability mask} The authors of \cite{sfmlearner} tackle this problem by having a CNN predict what pixels are valid to use in the photometric loss function. It shares the encoder of the pose predicting network but branches of into a different encoder which estimates a mask of the valid/explainable pixels. The loss function for the mask is the cross entropy loss compared to a mask filled with ones. The photometric loss function is augmented to include the explainability mask removing pixels that cannot be explained by the predicted depth and transformation. This encourages the mask to be filled with ones, but allows some slack due to pixels that can not be explained by the photometric loss.

\paragraph{Stationary pixels mask} The authors of \cite{monodepth2} introduced a mask to remove stationary pixels from the set of previous, current and next frame. This is done by creating a mask where the photometric error is smaller before applying the projection than after. This works because stationary pixels that have not moved in relation to the camera will of course have a small photometric loss without reprojection. This will remove pixels from the car dashboard and also nearby vehicles that are traveling at the same speed.